\subsection*{آ}
این گزاره \underline{\textbf{نادرست}} است.
برای آن یک مثال نقض می‌زنیم:

\setLTR

$
A =
\begin{bmatrix}
	1 & 2 & 3 & 4 \\
	4 & 1 & 2 & 3 \\
	3 & 4 & 1 & 2 \\
	2 & 3 & 4 & 1
\end{bmatrix}
, B = I = 
\begin{bmatrix}
	1 & 0 & 0 & 0 \\
	0 & 1 & 0 & 0 \\
	0 & 0 & 1 & 0 \\
	0 & 0 & 0 & 1
\end{bmatrix}
$

$C = AB = AI = A =
\begin{bmatrix}
	1 & 2 & 3 & 4 \\
	4 & 1 & 2 & 3 \\
	3 & 4 & 1 & 2 \\
	2 & 3 & 4 & 1
\end{bmatrix} \longrightarrow C_{3}=C_{1} +	C_{4} - C_{2}
$
\setRTL

اگر $C_{i}$ را برابر سطرهای ماتریس C بگیریم، با توجه به رابطه فوق، سطرهای این ماتریس مستقل خطی نیستند.

\subsection*{ب}

اگر در فضای برداری W بردارهای 
$w_{1},w_{2},w_{3},...,w_{n+1}$
فضای W را اسپن کنند، از آن‌جایی که این بردارها متمایز هستند و معادله‌ی 
$w_{1}+w_{2}+w_{3}+...+w_{n+1}=0$
برقرار است، پس بردارها وابسته خطی هستند.

پس می‌توان بردار 
$w_{n+1}$
را با استفاده از بقیه‌ی بردارها ساخت. درنتیجه با حذف این بردار، باز هم می‌توان فضای $W$ را اسپن کرد.

پس مجموعه‌ی بردارهای 
$w_{1},w_{2},...,w_{n}$
یک پایه برای فضای بردار فوق هستند و این بردارها مستقل خطی هستند زیرا برای اسپن فضای 
$W \in R^n$
به $n$ بردار نیاز داریم.

(می‌توان برای اثبات گزاره فوق از برهان خلف هم استفاده کرد و به تناقض رسید که برای اسپن فضای فوق از
$n-1$
بردار استفاده کرد که این یک تناقض است.)