می‌دانیم که ماتریس‌های مربعی برای وارون‌پذیر بودن باید دارای ستون‌های مستقل خطی باشند. با استفاده از برهان خلف فرض می‌کنیم این ستون‌ها مستقل خطی نیستند. یعنی 
$\lambda_1 ... \lambda_n$
وجود دارند به طوری که 
$\lambda_1A_1 + ... + \lambda_nA_n = 0$


می‌دانیم که:

\setLTR
$
\sum_{i=1}^{n}\sum_{j=1}^{n} \lambda_iA_{ji} = (1+..+n)(\lambda_1+...+\lambda_n)=0\longrightarrow
\lambda_1+...+\lambda_n = 0
$
\setRTL

فرض می‌کنیم که 
$\lambda_i$
مخالف صفر باشد و 
$i \neq n$
است.

حال با توجه به عبارت مقابل:

\setLTR
$
\lambda_1A_1+...+\lambda_{i-1}A_{i-1}+\lambda_{i}A_i+...\lambda_nA_n = \lambda_i
\begin{bmatrix}
	i \\
	i-1 \\
	\vdots \\
	n \\
	n-1 \\
	\vdots \\
	i +1 \\
\end{bmatrix} + ... + \lambda_n
\begin{bmatrix}
	n \\
	n-1 \\
	\vdots \\
	1
\end{bmatrix} = 0 \\ \xrightarrow{i , i+1} \begin{cases}
\lambda_i \times 1 + \lambda_{i+1} \times 2 + ... + + \lambda_{n} \times (n-i+1) = 0 \\
\lambda_i \times n +...+ \lambda_{n} \times (n-i) = 0
\end{cases} \xrightarrow{-} -n\lambda_i + \sum_{k=i}^{n} \lambda_k = 0 \xrightarrow{} -n\lambda_i =0 \\ \longrightarrow \lambda_i = 0
$
\setRTL


به تناقض رسیدیم که 
$\lambda_i = 0$
پس حکم اثبات می‌شود.