باید دو طرف قضیه را ثابت کنیم:

\begin{enumerate}
	\item  	اگر وابسته خطی باشند $\longleftarrow$ بردار صفر تولید می‌شود. 
	
	برای این اثبات از برهان خلف کمک می‌میگیریم:
	
	فرض می‌کنیم بردار صفر تولید نشده است و آخرین بردار در الگوریتم گرام‌اشمیت بردار 
	$q_m$
	بوده است، پس:
	\setLTR
	
	$\widetilde{q_m} = r_k - (q_1.r_k)q_1 - ... - (q_{k-1}.r_k)q_{k-1}=\frac{\widetilde{q_m}}{\|\widetilde{q_m}\|}$
	
	\setRTL
	
	طبق فرض خلف
	$q_i$
	ها مخالف صفر هستند و می‌دانیم که این الگوریتم بردارهای عمود تولید می‌کند، پس 
	$q_i$
	ها مستقل‌ خطی هستند. همچنین:
	
	\setLTR
	
	$q_1 = \frac{v_1}{\|v_1\|} \\ q_2 = \frac{v_2 - (q_1.v_2)q_1}{\|v_2 - (q_1v_2)q_1\|} = cv_2 - \frac{c(q_1v_2)}{\|v_1\|}v_1$
	
	\setRTL
	
	پس درنتیجه هر ترکیب خطی از 
	$q_i$
	ها را می‌توان از ترکیب خطی از 
	$r_i$
	ها نوشت. اما می‌دانیم که $q_i$ها m بردار مستقل خطی هستند و فضای $m$بعدی را span می‌کنند، پس $r_i$ها هم فضای $m$بعدی را span می‌کنند.
	اما می‌دانیم که $r_i$ها وابسته خطی هستند و نمی‌توانند فضای $m$بعدی را span کنند. پس به تناقض می‌سیم و حکم اثبات می‌شود. \\
	
	
	
	\item 	اگر بردار صفر تولید شود$r_i \longleftarrow$ ها وابسته خطی هستند.
	
	فرض می‌کنیم که $q_i$ برابر صفر است:
	
	\setLTR
	
	$r_i - \sum_{j=1}^{i-1}(q_j.r_i)q_j = 0$
	
	\setRTL
	
	همینطور می‌دانیم که:
	
		\setLTR
	
	$r_i = \sum_{j=1}^{i-1}(q_j.r_i)q_j$ // //
	$r_i = \sum_{j=1}^{mi-1}c_jr_j \longrightarrow r_i - c_{i-1}r_{i-1} - ... - c_1r_1 = 0$
	\setRTL
	
	که نشان میدهیم:	\setLTR
	
	$r_i - \sum_{j=1}^{i-1}(q_j.r_i)q_j = 0$
	
	\setRTL
	
	پس $r_i$ها وابسته خطی هستند.
	
	\end{enumerate}
	
	پس در نهایت هر دو طرف قضیه اثبات می‌شود.