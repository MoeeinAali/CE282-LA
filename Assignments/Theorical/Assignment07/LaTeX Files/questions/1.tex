با توجه به این که ماتریس‌های ما $Orthonormal$ نیستند، کمی ماتریس‌ها را تغییر می‌دهیم:

\setLTR
$
A = U \Sigma V^T
= \frac{1}{3} 
\begin{bmatrix}
2 & 2 & -1 \\	
x & 2 & 2 \\
2 & -1 & 2
\end{bmatrix}
\begin{bmatrix}
5 & 0\\
0 & \frac{5}{3} \\
0 & 0
\end{bmatrix}
\frac{1}{5}
\begin{bmatrix}
4 & -3 \\
3 & y
\end{bmatrix}
$
\setRTL

\subsection*{الف}

باتوجه به این که $U$ و $V$ دارای ستون‌های $Orthogonal$ هستند، داریم:

\setLTR
$
\begin{cases}
u_1.u_2=(2,x,2).(2,2,-1) = 2x+2 = 0 \longrightarrow x=-1 \\
v_1.v_2 = (4,-3).(3,y) = 12 - 3y = 0 \longrightarrow y =4 
\end{cases}
$
\setRTL

\subsubsection*{ب}

با توجه به این که ماتریس $\sigma$ در قطر اصلی خود دارای 2 مقدار ناصفر متمایز است، پس دارای رنک 2 است. 

همچنین مقدار ویژه‌ها برابر با مربع اعداد روی قطر اصلی هستند، پس مقادیر ویژه برابر هستند با $25$ و $\frac{25}{9}$.

برای بردار ویژه‌ی $AA^T$ نیز می‌توان یکی از ستون‌های ناصفر $U$ را مثال زد. به عنوان مثال: 
$(\frac{2}{3},\frac{2}{3},\frac{-1}{3})$

\subsection*{پ}

\setLTR
$
Av_i=\sigma_uU_i \longrightarrow A(cv_i) = c\sigma_iv_i \longrightarrow 
\begin{bmatrix}
	4 \\
	-3
\end{bmatrix} = 5v_1
\\
Av_1=\sigma_1U_1\longrightarrow A(5v_1)=5Av_1=5\sigma_1U_1 = 5\times5\times\frac{1}{3}\begin{bmatrix}
2 \\
-1 \\
2
\end{bmatrix} = \begin{bmatrix}
\frac{50}{3} \\
\frac{-25}{3} \\
\frac{50}{3	}
\end{bmatrix}
$

\setRTL

\subsection*{ث}

ماتریس را SVD کرده و سپس مقدار بیشینه را محاسبه می‌کنیم:

\setLTR
$
A = U\Sigma V^T \rightarrow \|Av\| = \| U\Sigma V^Tv\| = \sqrt[]{v^TV\Sigma^TU^TU\Sigma V^Tv} = \sqrt[]{(V^Tv)^T\Sigma^T\Sigma V^Tv} \\
\longrightarrow \sqrt[]{\sum_{i=n}^{n}((V^Tv)^T\Sigma^T\Sigma)_{1i}(V^Tv)_{i}} = \sqrt[]{\sum_{i=1}^{n}(V^Tv)^T_{i}\lambda_i(V^Tv)_{i}} = 
\sqrt[]{\sum_{i=1}^{n}\lambda_iv^TVV^Tv} = 
\sqrt[]{\sum_{i=1}^{n}\lambda_iv^T_iv_i} \\
= \sqrt[]{\sum_{i=1}^{n}\lambda_iv_i^2} \leq \sqrt[]{\sum_{i=1}^{n}\lambda_{max}v_i^2} =
= \sqrt[]{\lambda_{max}\sum_{i=1}^{n}v^2_i} = \sqrt[]{\lambda_{max}}\|v\| \\ \longrightarrow
\frac{\|Av\|}{\|v\|}\leq \frac{\sqrt[]{\lambda_{max}}\|v\|}{\|v\|} = \sqrt[]{\lambda_{max}} = \sigma_{max}
$
\setRTL

پس اگر $v$ما برابر با $e_i$ باشد:

\setLTR
$
\frac{\|Av\|}{\|v\|} = \sqrt[]{\sum_{i=1}^{n}\lambda_iv_i^2} = \sqrt[]{\lambda_i} = \sigma_i
$
\setRTL

پس این مقدار فقط به ازای 
$ke_i$
و 
$\sigma_{max}$
برابر با $\sigma_i$ رخ می‌دهد.

\setLTR
$
\sigma_{max} = 5 \ , \
v= \begin{bmatrix}
	1 \\
	0
\end{bmatrix}
$
\setRTL



\subsection*{ج}

\setLTR
$
A=U\Sigma V^T \longrightarrow A^+=V\Sigma^+U^T \\ \\
\Sigma^+ = \begin{bmatrix}
\frac{1}{5} & 0 & 0 \\
0 & \frac{3}{5} & 0
\end{bmatrix}\\
A^+ = \frac{1}{5}\begin{bmatrix}
	4 & 3 \\
	-3 & 4
\end{bmatrix} \times \frac{1}{5} \begin{bmatrix}
1 & 0 & 0 \\
0 & 3 & 0
\end{bmatrix}
\times \frac{1}{3} \begin{bmatrix}
2 & -1 &  2 \\
2 & 2 & -1 \\
-1 & 2 & 2 
\end{bmatrix} = 
\frac{1}{75} \begin{bmatrix}
26 & 14 & -1 \\
18 & 27 & -18  
\end{bmatrix}
$
\setRTL











