\subsubsection*{الف}{
از آن‌جایی که مجموع درایه‌های هرستون ماتریس $A$ برابر با $c$ است، پس مجموع‌ درایه‌های هر ستون ماتریس $A-cI$ برابر صفر است.

حال یک سطر از ماتریس را در نظر گرفته و باقی سطرها را به آن اضافه می‌‌کنیم بی‌آنکه دترمینان تغییر کند، می‌بینیم که این سطر صفر شده و مقدار دترمینان ماتریس $A-cI$ برابر صفر است. پس عدد $c$ مقدار ویژه است.

}
\subsubsection*{ب}{
برای اثبات این جکم، فرض می‌کنیم به ازای $n-1$ مستقل‌خطی هستند و با اضافه کردن $n$امین عدد همچنان مستقل‌خطی هستند. همچنین پایه ما $n=1$ است که بدیهتا مستقل‌خطی است.

برای اثبات استقرای فوق از برهان خلف استفاده می‌کنیم و فرض می‌کنیم که با اضافه کردن آخرین عدد دیگر مستقل خطی نیستند!  پس داریم:

\setLTR
$
\sum_{i=1}^{n-1}c_ie^{\lambda_ix} = e^{\lambda_nx} \xrightarrow{'} \sum_{i=1}^{n-1}c_i\lambda_ie^{\lambda_ix} = \lambda_ne^{\lambda_nx} \\  \\
\sum_{i=1}^{n-1}c_i\lambda_ne^{\lambda_ix} = \lambda_n\sum_{i=1}^{n-1}c_ie^{\lambda_ix}=\lambda_ne^{\lambda_nx}
$
\setRTL

حال دو رابطه فوق را از هم کم کرده:

\setLTR
$
\sum_{i=1}^{n-1}c_i\lambda_ie^{\lambda_ix} - \sum_{i=1}^{n-1}c_ie^{\lambda_ix} = 0 \longrightarrow 
\sum_{i=1}^{n-1}c_i(\lambda_i-\lambda_n)e^{\lambda_ix} = 0
$
\setRTL

حال می‌دانیم که به دلیل مستقل‌خطی بودن باید ضرایب صفر شوند تا رابطه فوق برقرار باشد. همچنین می‌دانیم که $\lambda_n$ با بقیه برابر نیست و آن پرانتز صفر نمی‌شود. پس باید $e^{\lambda_nx}$ برابر صفر شود که تناقض است و حکم مسئله اثبات می‌شود.

}