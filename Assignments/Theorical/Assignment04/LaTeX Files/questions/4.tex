باید ثابت کنیم که:

\setLTR
$
\exists \ S \in \mathcal{L}(V,W) \longleftrightarrow Range(T_1) = Range(T_2) \\ 
$
\setRTL

ابتدا طرف اول را ثابت می‌کنیم. اگر S وجود داشته باشد، چون 
$T_1 = T_2 S$
است، پس هر عضو از 
$Range(T_1)$
عضو 
$Range(T_2)$
هم است. چون که به ازای هر 
$u$
دلخواه که به 
$T_1$
داده می	‌میشود، یک
$v=Su$
هم وجود دارد که به ازای آن خروجی 
$T_2$
با خروجی 
$T_1$
برابر خواهد شد. پس:

\setLTR
$
Range(T_1) \subseteq Range(T_2)
$
\setRTL

از آن‌جایی که 
$S$
معکوس‌پذیر است، پس:

\setLTR
$
T_1 = T_2 S \longrightarrow T_1S^{-1} = T_2SS^{-1} \longrightarrow T_1S^{-1}=T_2
$
\setRTL

پس مشابه بالا می‌توان گفت:

\setLTR
$
Range(T_2) \subseteq Range(T_1)
$
\setRTL

در نهایت طرف اول حکم اثبات می‌شود:

\setLTR
$
\begin{cases}
	Range(T_2) \subseteq Range(T_1) \\
	Range(T_1) \subseteq Range(T_2)
	
\end{cases} \longrightarrow Range(T_1) = Range(T_2)
$
\setRTL

حال طرف دیگر حکم را ثابت می‌کنیم. ما S را این‌گونه تعریف می‌کنیم که اگر 
$e_1 , e_2 , e_3 ,...,e_n$
پایه در 
$T_1$
باشد، آن‌گاه 
$e'_1 , e'_2 , e'_3 ,...,e'_n$
پایه در 
$T_2$
باشند، به طوری که:
$Se_i=e'_i$

پس در نتیجه تمامی عناصر موجود در
$T_2$
ساخته می‌شوند.

\setLTR
$
\alpha \in T_2 \longrightarrow \alpha = \alpha_1e'_1 + \alpha_2e'_2 +...+\alpha_ne'_n \longrightarrow \\ \alpha_1Se_1 + ... + \alpha_nSe_n = S\alpha_1e_1 + ... + S\alpha_ne_n = S(\alpha_1e_1 + ... + \alpha_ne_n)
$
\setRTL

یک ترکیب خطی ساخته شد که به 
$\alpha$
می‌رود. تبدیل 
$S$
وارون‌پذیر است، چون یک‌به‌یک است. یک‌به‌یک بودن را با برهان خلف ثابت می‌کنیم. فرض می‌کنیم 
$u \neq v , S(u) = S(v)$
باشد، پس داریم:

\setLTR
$
S(u-v) = 0  \\ S(0) = 0
$
\setRTL

حال باید نشان دهیم که 
$u = v$
تا به تناقض بخوریم.

\setLTR
$
\begin{cases}
S(u) = \alpha_1e'_1 , \alpha_2e'_2 , e_3 ,...,\alpha_ne'_n = S(\alpha_1e_1 + ... + \alpha_ne_n) \\
S(v) = \beta_1e'_1 , \beta_2e'_2 , e_3 ,...,\beta_ne'_n = S(\beta_1e_1 + ... + \beta_ne_n)
\end{cases} \rightarrow (\alpha_1-\beta_1)e'_1 + ... + (\alpha_n-\beta_n)e'_n =0
$
\setRTL

طبق تعریف پایه بودن 
$e'_1 , ... , e'_n$
، پس ضرایب برابر صفر هستند و در نتیجه:

\setLTR
$
\alpha_i - \beta_i =0 \longrightarrow \alpha_i = \beta_i \longrightarrow u = v
$
\setRTL











