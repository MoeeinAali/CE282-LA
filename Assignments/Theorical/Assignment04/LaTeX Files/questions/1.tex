\subsection*{الف}


اگر A یک ماتریس mدرn و B یک ماتریس mدرk باشد، بنابراین AB یک ماتریس nدرk خواهد بود، پس:

\setLTR
$
\begin{cases}
	Nullity(A) + Rank(A) = m 
	Nullity(B) + Rank(B) = k \\
	Nullity(AB) + Rank(AB) = k
\end{cases}
$
\setRTL

با توجه به معادلات بالا:

\setLTR
$
	Nullity(B) + Rank(B) = 	Nullity(AB) + Rank(AB) \\ \longrightarrow Nullity(B) + Rank(B) + Nullity(A) + Rank(A) = Nullity(AB) + Rank(AB) + m
$
\setRTL

همچنین می‌دانیم که:

\setLTR
$
Rank(AB) \leq Rank(B)
$
\setRTL

پس درنتیجه:

\setLTR
$
\begin{cases}
	Nullity(B) + Nullity(A) + Rank(A) \geq Nullity(AB) + m \\
	Rank(A) \leq m
\end{cases} \longrightarrow 
Nullity(A) + Nullity(B) \ge Nullity(AB)
$
\setRTL

درنهایت خواسته اصلی سوال را اثبات می‌کنیم:

\setLTR
$
Nullity(B) + Nullity(C) \ge  Nullity(BC) \\ \longrightarrow  Nullity(A) + Nullity(BC) \leq Nullity(A) + Nullity(B) + Nullity(C) \\ \longrightarrow  Nullity(A(BC)) \leq Nullity(A) + Nullity(B) + Nullity(C) \\ \longrightarrow Nullity(ABC) \leq Nullity(A) + Nullity(B) + Nullity(C)
$
\setRTL
\subsection*{ب}

طبق قضیه‌های از قبل ثابت شده می‌دانیم که:

\setLTR
\begin{enumerate}[label=(\roman*)]
	\item $Rank\{A^T\} = Rank{A}$
	\item $Rank\{AB\} \leq Rank{A}$
\end{enumerate}
\setRTL

حال نامساوی خواسته شده را اثبات ‌می‌کنیم:

\setLTR

$Rank\{AB\} = Rank\{(AB)^T\} = Rank\{B^TA^T\} \leq Rank\{B^T\} = Rank\{B\} \\ 
\longrightarrow
\begin{cases}
	Rank\{AB\} \leq Rank\{A\} \\
	Rank\{AB\} \leq Rank\{B\}
\end{cases} \longrightarrow Rank\{AB\} \leq min(Rank\{A\},Rank\{B\}) \\ \\ \\
Rank\{ABC\}  \xrightarrow[\text{اثبات بالا}]{
	BC = D
} Rank\{AD\} \leq min(Rank\{A\},Rank\{D\}) \\
\xrightarrow[]{
	Rank\{D\} \leq min(Rank\{C\},Rank\{B\})
} 
Rank\{ABC\} \leq min(Rank\{A\} , Rank\{B\}, Rank\{C\})
$
\setRTL

\subsection*{پ}
در ابتدا می‌دانیم که:

\setLTR

$
\begin{cases}
	ABC = 0 \rightarrow Nullity\{ABC\} = n \\ 
	Nullity\{ABC\} \leq Nullity\{A\} + Nullity\{B\} + Nullity\{C\}
\end{cases}
$
\setRTL

همچنین:
\setLTR

$
Rank\{A\} + Nullity\{A\} = n 
$
\setRTL

همینطور اگر فرض کنیم که
$Nullity\{C\} , Nullity\{B\} \leq Nullity\{A\}$
است، پس داریم:

\setLTR

$
n \leq 3 \times Nullity\{A\} \rightarrow \frac{n}{3} \leq Nullity\{A\} \rightarrow 
Rank(A) \leq \frac{2n}{3} \\ 
Rank\{CBA\} \leq min(Rank\{A\},Rank\{B\},Rank\{C\}) \leq Rank\{A\} \leq \frac{2n}{3}
$

\setRTL

\subsection*{ت}
برای اثبات حکم مسئله ما باید این 3 رابطه را اثبات کنیم:

\setLTR
\begin{itemize}
	\item $N(A) = N(A^2) \longrightarrow Range(A) = Range(A^2)$ 
	\item $ Range(A) = Range(A^2) \longrightarrow Range(A) \cap N(A) = \{0\}$ 
	\item $Range(A) \cap N(A) = \{0\} \longrightarrow N(A) = N(A^2) $ 
\end{itemize}
\setRTL

برای اثبات \textbf{حکم اول} داریم:

\setLTR
$
N(A) = N(A^2) \longrightarrow Nullity(A) = Nullity(A^2) \longrightarrow n - Nullity(A) = n - Nullity(A^2)  \longrightarrow  \\ Rank(A) = Rank(A^2) 
ColumnSpace(AB) \subseteq ColumnSpace(A) \longrightarrow ColumnSpace(A^2) \subseteq ColumnSpace(A)
$
\setRTL

می‌دانیم که Rank این دو با هم برابر است، پس می‌توان گفت که اندازه پایه هردوی آن‌ها برابر است، پس پایه‌ی فضای ستونی 
$A$
 برابر با پایه‌ی فضای ستونی 
$A^2$
است. پس:

\setLTR
$
ColumnSpace(A) = ColumnSpace(A^2) \longrightarrow Range(A) = Range(A^2)
$
\setRTL

حال برای اثبات \textbf{حکم دومی} از برهان خلف استفاده کرده و فرض می‌کنیم که بردار غیر صفری مثل u وجود دارد که در هر دو وجود دارد و چون u عضوی از فضای ستونی A است، پس برداری همانند v هم وجود دارد که 
$Av=u$
باشد. پس:

\setLTR
$
u \in N(A) \cap Range(A) \rightarrow u \in N(A) \rightarrow Au = 0 \rightarrow AAv = 0 \rightarrow A^2v=0
$
\setRTL

پس v عضوی از NullSpace فضای $A$ است و نه فضای 
$A^2$
.
همچنین می‌دانیم که:

\setLTR
$
NullSpace(A) \subseteq NullSpace(A^2)
$
\setRTL

پس:

\setLTR
$
Nullity(A^2) > Nullity(A) \rightarrow n - Nullity(A^2) < n - Nullity(A) \rightarrow Rank(A^2) < Rank(A) 
$
\setRTL

پس در نتیجه 
$Range(A) \neq Range(A^2)$
است، پس این تناقض است و قطعا اشتراک این دو برابر با 
$\{0\}$
است.


در نهایت باید \textbf{حکم سوم} را اثبات کنیم. می‌دانیم که
$N(A) \subseteq N(A^2)$
است، پس طبق برهان خلف فرض می‌کنیم که برداری مانند u وجود دارد که در 
$N(A^2)$
باشد اما در
$N(A)$
نباشد. پس:

\setLTR
$
AAu = 0 \longrightarrow A(Au) = 0
$
\setRTL

پس بردار 
$Au$
عضو 
$N(A)$
است و همچنین عضو 
$ColumnSpace(A)$
نیز هست. در نتیجه عضو 
$N(A) \cap R(A)$
هم هست. و چون طبق بخش قبل اشتراک این دو فقط شامل بردار صفر است:

\setLTR
$
Au =0 \longrightarrow u \in N(A) 
$
\setRTL

پس با توجه به تناقض به وجود آمده، حکم به کمک برهان خلف اثبات شد.















