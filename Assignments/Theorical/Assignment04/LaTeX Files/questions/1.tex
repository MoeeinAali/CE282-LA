\subsubsection*{الف}

این گزاره \underline{\textbf{نادرست}} است.

زیرا بردار $O$ و هر بردار دیگری بر هم عمود هستند ولی مستقل خطی نیستند!


\subsubsection*{ب}

این گزاره \underline{\textbf{درست}} است.

\setLTR

$\begin{cases}
	\| u - v \| ^ 2 =  (u - v)^T(u-v) = u^Tv-v^Tu+v^Tv=u^Tu-2u^Tv+v^Tv = \| u \| - 2 \langle  u,v\rangle  + \|v \| \\
		\| u - v \| ^ 2 = 	\| u  \| ^ 2 + 	\|  v \| ^ 2
\end{cases}$

\setRTL

پس درنتیجه
 $\| u \| + \| v \| = 0$
  و دو بردار بر هم عمود هستند.

\subsubsection*{ج}
این گزاره \underline{\textbf{درست}} است.

\setLTR

$\begin{cases}
	\langle u-v,u-v\rangle = \| u \| ^ 2 + \| v \| ^ 2 - 2 \langle u,v\rangle \\
	 \|v\| = \|v\| = 1 \\
	 \langle u,v \rangle = 1
\end{cases}  \longrightarrow  \langle u -v  , u-v \rangle = 1^2 + 1^2 -2(1) = 0 \rightarrow u - v = 0 \rightarrow u = v$

\setRTL

