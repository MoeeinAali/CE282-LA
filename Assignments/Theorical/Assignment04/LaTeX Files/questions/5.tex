

\setLTR
$
A^2 = AB + 2I  = A(A-B) = 2I \longrightarrow Dim(2I)=3 \longrightarrow Dim(A(A-B))=3 \\
Dim(A(A-B))=3 \leq Dim(A) , Dim(A-B) \leq 3 \longrightarrow Dim(A) = Dim(A-B) = 3
$
\setRTL

با توجه به این که سطرها و ستون‌های A , B مستقل خطی هستند، پس دو ماتریس A و A-B وارون پذیر هستند.
درنتیجه 
 $A^{-1}$
 وجود دارد. پس دو طرف معادله فوق را در آن ضرب کرده و داریم:
 
 \setLTR
 $
A(A-B)=2I \xrightarrow[]{\times A^{-1}}
A-B=2A^{-1} \longrightarrow A-2A^{-1} = B
 $
 \setRTL
 
 معادله فوق ممکنه است جواب داشته باشد یا ممکن است جواب نداشته باشد!
 
  \setLTR
 $
 A= \begin{bmatrix}
 	a & b & c \\ d & e & f \\ g & h & i
 	\end{bmatrix} \longrightarrow
 	 A^{-1}= \frac{1}{
  a(ei-fh)-b(di-fg)+(dh-eg)
  }\begin{bmatrix}
  ei-fh & ch-bi & bf-ce \\
  fg-di & cg-ai & cd-af \\
  dh-eg & bg-ah & ae-bd
  \end{bmatrix} \\ \\ \\
 \begin{cases}
 	 A-2A^{-1} = \begin{bmatrix}
 		a - \frac{2(ei-fh)}{a(ei-fh)-b(di-fg)+(dh-eg)} &
 		b - \frac{2(ch-bi)}{a(ei-fh)-b(di-fg)+(dh-eg)} &
 		c - \frac{2(bf-ce)}{a(ei-fh)-b(di-fg)+(dh-eg)} \\
 		d - \frac{2(fg-di)}{a(ei-fh)-b(di-fg)+(dh-eg)} &
 		e - \frac{2(cg-ai)}{a(ei-fh)-b(di-fg)+(dh-eg)} &
 		f - \frac{2(cd-af)}{a(ei-fh)-b(di-fg)+(dh-eg)} \\
 		g - \frac{2(dh-eg)}{a(ei-fh)-b(di-fg)+(dh-eg)} &
 		h - \frac{2(bg-ah)}{a(ei-fh)-b(di-fg)+(dh-eg)} &
 		i - \frac{2(ae-bd)}{a(ei-fh)-b(di-fg)+(dh-eg)} \\
 	\end{bmatrix}
 	\\ B= \begin{bmatrix}
 		-1 & 1 & -1 \\ 0 & -1 & 0 \\ 1 & 2 & -2
 	\end{bmatrix}
 	\end{cases}
 $
 
 $
 \begin{cases}
 	 a - \frac{2(ei-fh)}{a(ei-fh)-b(di-fg)+(dh-eg)} = -1 \\ \\
 	b - \frac{2(ch-bi)}{a(ei-fh)-b(di-fg)+(dh-eg)} = 1 \\ \\ 
 	c - \frac{2(bf-ce)}{a(ei-fh)-b(di-fg)+(dh-eg)} = -1 \\ \\
 	d - \frac{2(fg-di)}{a(ei-fh)-b(di-fg)+(dh-eg)} = 0 \\ \\ 
 	e - \frac{2(cg-ai)}{a(ei-fh)-b(di-fg)+(dh-eg)} = -1 \\ \\ 
 	f - \frac{2(cd-af)}{a(ei-fh)-b(di-fg)+(dh-eg)} = 0 \\ \\ 
 	g - \frac{2(dh-eg)}{a(ei-fh)-b(di-fg)+(dh-eg)} = 1 \\ \\ 
 	h - \frac{2(bg-ah)}{a(ei-fh)-b(di-fg)+(dh-eg)} = 2 \\ \\
 	i - \frac{2(ae-bd)}{a(ei-fh)-b(di-fg)+(dh-eg)} = -2
 	\end{cases}
 $
 
 
 \setRTL
 
 حال معادلات فوق را حل می‌کنیم و مجهول‌ها را بدست می‌آوریم.