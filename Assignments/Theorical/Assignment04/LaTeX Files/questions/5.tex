ابتدا باید ثابت کنیم که اگر $b=c=0$ باشد، آنگاه $T$ یک تبدیل خطی است.
پس:
\setLTR

$T(p(x))=(3p(4)+5p'(6),\int_{-1}^{2}x^3p(x)dx)$

\setRTL

پس باید دو ویژگی زیر را ثابت کنیم:

\setLTR

\begin{enumerate}
	\item $T((p+q)(x))=(3p(4)+3q(4)+5p'(6)+5q'(6),
	\int_{-1}^{2}x^3p(x)dx + \int_{-1}^{2}x^3q(x)dx) = T(p(x))+T(q(x))$
	
	\item $T(\alpha P(x)) = (3\alpha P(4) + 5\alpha P'(6) , \int_{-1}^{2}\alpha x^3P(x)dx)=\alpha(3P(4)+5P'(6),\int_{-1}^{2}x^3P(x)dx)=\alpha T(P(x))$
	\end{enumerate}
	
	\setRTL
	حال ثابت می‌کنیم که اگر T تبدیل خطی باشد، آنگاه باید
	$b=c=0$
	باشد. پس:
	
	\setLTR
	
\begin{itemize}
	\item $3\alpha p(4) + 5\alpha p'(6) + \beta\alpha ^ 2 p(1)p(2) = \alpha (3p(4)+5p'(6)+\beta p(1)p(2))$
	
	$ \longrightarrow \beta\alpha^2p(1)p(2)=\beta\alpha p(1)p(2)\longrightarrow \beta\alpha = \beta \longrightarrow \beta = 0$
	
	\item $\int_{-1}^{2}x^3\alpha p(x)dx + c\sin(\alpha p(0))=\alpha(\int_{-1}^{2}x^3p(x)dx+c\sin(p(0)))$
	
	$\longrightarrow c\sin(\alpha p(0))=\alpha c\sin(p(0))\longrightarrow c = 0$
\end{itemize}

\setRTL

عبارات فوق به ازای هر $\alpha$ و $p$ همواره برقرار است و حکم ثابت می‌شود.







	
	
	
	
	
	
	
	
	
	
	
	