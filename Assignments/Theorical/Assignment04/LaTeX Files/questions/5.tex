\setLTR
$
A^2 = AB + 2I  = A(A-B) = 2I \longrightarrow A(\frac{A-B}{2}) = I
$
\setRTL

باید ماتریس A وارون‌پذیر باشد، پس:

\setLTR
$
A^2 = AB + 2I \longrightarrow A = B + 2A^{-1} \longrightarrow B = A - 2A^{-1} \\ \longrightarrow BA = (A-2A^{-1})A = A^2 -2I \longrightarrow A^2 = BA + 2I \\ \\
\begin{cases}
	A^2 = AB + 2I  \\
	A^2 = BA + 2I
\end{cases} $\longrightarrow$ AB = BA
$

\setRTL


حال فرض می‌کنیم که 
$Y = 2A - B$
است، پس:

\setLTR
$
Y ^ 2 = (2A - B)^2 = 4A^2 - 4AB + B^2 = 4(A^2 -AB) + B^2 \xrightarrow{A^2 = AB + 2I} Y^2 = 8I + B^2 \\ \\
Y^2 = \begin{bmatrix}
	8 & 0 & 0 \\
	0 & 8 & 0 \\
	0 & 0 & 8
\end{bmatrix} + 
\begin{bmatrix}
	0 & -4 & 3 \\
	0 & 1 & 0 \\
	-3 & -5 & 3 
\end{bmatrix} = 
\begin{bmatrix}
	8 & -4 & 3 \\
	0 & 9 & 0 \\
	-3 & -5 & 11
\end{bmatrix} \\ \\
|Y|^2 = |Y^2| =872 \\ \\ 
|2A-B| = |Y| = \sqrt[]{872} \simeq 29.5
$
\setRTL

طبق فرض سوال می‌دانیم که درایه‌های ماتریس
$A$
صحیح هستند، پس درایه‌های ماتریس 
$2A-B$
هم صحیح هستند، در نتیجه دترمینان 
$Y$
هم باید صحیح باشد، اما نیست !

پس ماتریس 
$َA$
وجود ندارد.
